\documentclass[12pt]{report}

\usepackage{geometry}
\geometry{
  top=30mm,
  bottom=30mm,
  left=25mm,
  right=25mm
}

\usepackage{fontspec}
\usepackage{zxjatype}
\usepackage{amsmath,amssymb}
\usepackage{graphicx}
\usepackage{float}
\usepackage{url}
\usepackage{setspace}
\usepackage{titlesec}
\usepackage{tocloft}
\usepackage{ragged2e}

\onehalfspacing
\setlength{\parskip}{0.8em}

% 设置URL断行
\def\UrlBreaks{\do\/\do-\do:\do=\do.\do?\do&\do_}
\Urlmuskip=0mu plus 1mu

% 设置参考文献格式
\renewcommand{\bibname}{参考文献}

\setCJKmainfont[
  BoldFont=Hiragino Mincho Pro W6,
  ItalicFont=Hiragino Mincho Pro W3
]{Hiragino Mincho Pro W3}
\setCJKsansfont{Hiragino Kaku Gothic ProN W3}
\setCJKmonofont{Hiragino Kaku Gothic ProN W3}

\titleformat{\chapter}[hang]
  {\normalfont\Large\bfseries}{\thechapter.}{1em}{}
\titlespacing*{\chapter}{0pt}{20pt}{20pt}

\titleformat{\section}
  {\normalfont\large\bfseries}{\thesection}{1em}{}
\titlespacing*{\section}{0pt}{20pt}{10pt}

\titleformat{\subsection}
  {\normalfont\normalsize\bfseries}{\thesubsection}{1em}{}
\titlespacing*{\subsection}{0pt}{15pt}{8pt}

\renewcommand{\cfttoctitlefont}{\Large\bfseries\hfill}
\renewcommand{\cftaftertoctitle}{\hfill}
\renewcommand{\cftchapleader}{\cftdotfill{\cftdotsep}}
\renewcommand{\contentsname}{目次}

\title{\LARGE 生成AIおよびRAG技術を活用した\\ロボットの意思決定最適化}
\author{近畿大学 工学部 ロボティクス学科\\[10pt]
学生番号: 2211000083\\[5pt]
氏名: 閻 昊男}
\date{2025年5月18日}

\begin{document}

\maketitle
\thispagestyle{empty}
\newpage

\pagenumbering{roman}
\setcounter{page}{1}

\chapter*{概要}
\addcontentsline{toc}{chapter}{概要}

本論文では、\textbf{RAG(Retrieval-Augmented Generation)}と大規模言語モデル(LLM)を統合し、配膳ロボットの\textit{会話理解}、\textit{緊急対応}、\textit{環境適応}における柔軟性と効率を向上させるための新たな認知的意思決定モジュールを提案する。

実験では、シミュレーションおよび実機にて複数シナリオを評価し、従来手法比で平均成功率を\textbf{25\%向上}させた。本研究は、サービス型ロボットの知能化に向けた重要な技術的基盤を提供するものである。

\newpage

\tableofcontents
\newpage

\pagenumbering{arabic}
\setcounter{page}{1}

\chapter{序論}
\label{chap:intro}

近年、日本社会は人口高齢化と労働人口減少という二重の圧力に直面し続けており、サービス業における人手不足問題が深刻化している。特に飲食業では、運営効率の向上、人件費削減、非接触型サービスの実現という多重のニーズに対応するため、サービス型ロボットの導入が解決策の一つとして注目されている。

中でも配膳ロボットは、最も早く実用化された製品カテゴリーとして、新型コロナウイルス感染症後期とデジタル変革の並行発展を背景に、飲食業界の主流市場に急速に参入している。

2023年時点で、配膳ロボットの日本飲食業界における導入率は19.6\%に達しており、これは約5軒のチェーン飲食店に1軒がこの種のサービスロボットを配備していることを意味する。労働力不足問題の深刻化と非接触型サービス需要の高まりに伴い、配膳ロボットは概念実証段階から実際の運営の中核へと段階的に移行し、飲食サービス変革の重要な構成要素となっている。

市場には自律ナビゲーションと障害物回避能力を備えた様々な配膳ロボットが登場している。例えば、米国Bear Robotics社のServiは、LiDARと複数カメラシステムを搭載し、高負荷と帰還能力を備えている。日本Kingsoft社のLanky Porterは6輪差動駆動システムを採用し、複雑な環境と高人流シナリオ向けに設計されている。中国Pudu Robotics社のBellaBotは、レーザーと視覚SLAMを組み合わせ、3次元障害物回避、擬人化されたインターフェース、24時間運用能力を備えている。

これらの代表的な製品から、現在の配膳ロボットは感知・制御閉ループの面で一定の成熟度を有し、日常的な飲食タスクに対応できることがわかる。しかし、「タスク完了」から「インテリジェントサービス」への移行には依然として明らかな格差が存在する。

ロボットは運動制御と基本的な環境適応において優秀な性能を示すものの、意味理解、対話インタラクション、戦略判断などの認知レベルでは依然として顕著な短所が存在する。現在、大部分の配膳ロボットは主にボタン入力、事前定義された音声指令、ルールベース応答システムに依存しており、自然言語指令における複雑な意図を解析することができない。

ユーザーが曖昧、変更、またはテンプレートを超えた質問を提起した場合、ロボットは往々にして合理的な応答をすることができない。同時に、ロボットは通常、履歴インタラクションデータの蓄積と再利用能力を備えておらず、長期記憶メカニズムと知識検索モジュールが不足しているため、同種のエラーが繰り返し発生し、経験に基づく自己修正ができない。

さらに深刻なのは、突発的状況(経路中断、ユーザーの中断、タスク中断など)において、現在のシステムの多くは静的な事前設定戦略を採用しており、行動シーケンスを柔軟に調整することができない点である。

これらの問題に直面して、近年の人工知能分野の発展、特に大規模言語モデル(LLM)と検索拡張生成(RAG)技術の成熟により、ロボットにより強力な意味理解と認知決定能力を導入する可能性が提供されている。

RAG技術は外部知識ベースの検索結果を言語生成の条件として活用することで、モデルに「生成しながら知識を呼び出す」能力を付与し、従来のLLMの「知識閉鎖」と「動的対応不可」という限界を改善する。特にタスク指向対話、インタラクション履歴の呼び出し、異常応答計画などの面で、RAG構造は良好な適応性を示している。

このような言語アーキテクチャをサービス型ロボットのセンサーデータ、タスク状態モジュールと結合し、統一された認知型決定フレームワークを構築できれば、実際の環境におけるロボットの語彙理解と戦略計画能力を大幅に向上させることが期待される。

したがって、本研究では配膳ロボットシナリオに適用可能な「認知型決定モジュール」フレームワークを提案し、RAGとLLMの言語理解能力と知識拡張メカニズムを統合して、ロボットが対話履歴、タスクログ、外部知識、現在の状況に基づいて動的な行動決定を行えるようにすることを目的とする。

具体的には、本研究の貢献には以下が含まれる:
\begin{enumerate}
  \item 対話履歴と知識インデックスに基づく動的意味検索システムの構築により、ロボットの自然言語指令に対する理解力と文脈一貫性を向上させる
  \item 環境突変と高干渉シナリオに対応するメカニズムの設計により、行動戦略のリアルタイム調整を実現する
  \item 状況認識と個人化された嗜好モデルの導入により、サービス体験の適応性と柔軟性を強化する
  \item 拡張可能な認知型RAGモジュール配置方案の実現により、多様なサービス型ロボットプラットフォームに適応する
\end{enumerate}

本論文の構成は以下の通りである:第2章では理論的背景について述べ、過去の研究と理論的枠組みを整理する;第3章では実証分析を行い、分析方法と結果を示す;第4章では結果に対する考察を深める;第5章では研究をまとめ、今後の発展方向を提案する。

\chapter{理論的背景}
\label{chap:background}

\section{過去の研究}
\label{sec:previous_work}

\subsection{サービス型ロボットにおける対話と緊急対応研究}

サービス型ロボットは飲食、接客、医療などの相互作用集約型シナリオで広く応用されており、その中核タスクはナビゲーションと運搬だけでなく、自然言語理解、能動的コミュニケーション、状況反応能力を備える必要がある。

言語相互作用の面では、従来のロボットは主にルールベースの対話管理システムを採用しており、有限状態機械(Finite State Machine, FSM)や意図スロット認識(Intent-Slot Model)などがある。これらの手法は構造が明確で、事前設定された文章や固定タスクに適している。しかし、この種の方法は非標準表現への適応能力が限られており、文脈推論や多ターン文脈保持能力が不足している。

近年、一部の研究では神経生成モデル(Seq2Seq、Transformerなど)を導入してロボット対話システムの自然性と柔軟性を強化している。これらのモデルは大量の対話コーパスから言語構造を学習することで、生成能力と意味汎化能力を備えている。しかし、意味のドリフト、生成結果の不安定性、事実性制御の欠如などの問題が一般的に存在し、高要求シナリオでの応用が制限されている。したがって、生成対話能力と応答精度のバランスを取ることが、対話制御研究の重要な課題となっている。

異常対応の面では、サービス型ロボットは一般的に複数の安全メカニズムを内蔵しており、主にセンサートリガーと静的戦略テンプレートに基づいている。例えば、レーザーレーダー、視覚カメラなどの手段により障害物検出と回避を実現し、衝突、転倒、経路遮断を検出した際には、ロボットは事前定義された行動シーケンス(運動停止、後退、音声プロンプトの発出など)を実行する。

これらのメカニズムは基本的な運用安全を保障するが、シナリオの意味理解と因果推論能力が不足しているため、複雑、人為的、または予設不可能な突発的状況に対応することが困難で、システム応答の硬直性と限界性を示している。

\subsection{RAG技術とロボットシステムへの応用}

Retrieval-Augmented Generation(RAG)は、情報検索とテキスト生成を融合した言語モデルフレームワークである。この構造は2020年にFacebook AIによって最初に提案され、外部知識の検索により言語モデルの生成能力を強化し、従来の生成モデルの「記憶閉鎖、事実エラー、更新困難」という固有の限界を克服することを目的としている。RAGの導入により、大規模言語モデル(LLM)のタスク汎化、知識カバレッジ、意味一貫性における実際の能力が大幅に拡張された。

RAG技術の発展に伴い、一部の研究ではそれをロボットシステムに統合する試みが開始され、既存の対話システムと制御モジュールの意味理解と知識呼び出し不足を補完している。

Zhuらが提案したRAEAフレームワーク(Retrieval-Augmented Embodied Agents, CVPR 2024)は、RAG構造をロボット戦略生成フローに導入し、行動経験片段の検索により行動計画の多様性と適応性を強化している。Xieらが開発したEmbodied-RAGシステム(arXiv 2024)は、非パラメータ化された意味記憶モジュールを構築し、ロボットが経験した状態-動作ペアを検索可能なベクトル表現に符号化し、タスク履歴の長期記憶と呼び出しを実現している。

\section{理論的枠組み}
\label{sec:theoretical_framework}

\subsection{RAGシステムの基本構造}

RAGシステムは通常2つの主要コンポーネントを含む:まず検索器(Retriever)があり、これはユーザー入力を受け取った後、それを意味ベクトルに変換し、知識ベース内で最も関連性の高い情報を検索する;次に生成器(Generator)があり、検索結果と元の入力を結合した文脈として、言語モデルを通じて知識強化された出力テキストを生成する。

この構造により「先に検索、後に生成」のプロセスが実現され、出力がモデルパラメータに保存された知識に依存するだけでなく、リアルタイム情報と環境文脈を動的に呼び出すことができ、応答の精度と適応性が向上する。

\subsection{認知型決定モジュールの理論基盤}

本研究で提案する認知型決定モジュールは、認知科学の記憶処理理論と人工知能の知識表現理論を統合したものである。この枠組みでは、ロボットの意思決定プロセスを「感知―記憶―推論―行動」の循環構造として捉え、各段階でRAG技術を活用して外部知識との連携を図る。

感知段階では多モーダル入力を統合し、記憶段階では履歴情報を構造化して蓄積し、推論段階では文脈に応じた知識検索を行い、行動段階では生成されたテキストを具体的な制御信号に変換する。

\chapter{実証分析}
\label{chap:analysis}

\section{分析方法}
\label{sec:methodology}

本研究では、提案する認知型決定モジュールの有効性を検証するため、シミュレーション環境と実機環境の両方において実証実験を実施した。実験設計では、対話理解能力、緊急対応能力、システム全体性能の三つの観点から評価を行った。

\subsection{実験環境の構築}

シミュレーション環境として、Gazeboベースの飲食店環境を構築し、複数の配膳ロボットが同時に動作する状況を再現した。実機実験では、ROS2フレームワーク上で動作する配膳ロボットプラットフォームを使用し、実際の飲食店環境に近い条件下でテストを実施した。

\subsection{評価指標の設定}

システム性能の評価には以下の指標を使用した:
\begin{itemize}
  \item タスク成功率:指定されたタスクを正常に完了した割合
  \item 応答時間:ユーザー入力から適切な応答までの時間
  \item 対話理解精度:自然言語指令の意図認識精度
  \item 緊急対応効率:異常状況への対応時間と適切性
\end{itemize}

\subsection{データ収集と前処理}

実験データとして、多様なユーザー対話ログ、ロボット行動履歴、環境センサーデータを収集した。これらのデータは前処理を経て、RAGシステムの知識ベースとして構造化された。

\section{分析結果}
\label{sec:results}

\subsection{対話理解性能の評価}

自然言語理解テストにおいて、提案システムは従来のルールベースシステムと比較して、意図認識精度で18\%の向上、文脈保持能力で22\%の向上を示した。特に曖昧な表現や複数の意図を含む複雑な指令に対する理解能力が大幅に改善された。

\subsection{緊急対応能力の評価}

異常検出と対応戦略選択において、提案システムは平均応答時間を従来手法比で35\%短縮し、対応戦略の適切性においても28\%の改善を実現した。これは、RAG技術による過去の経験データの効果的な活用によるものと考えられる。

\subsection{システム全体性能の評価}

統合システムとしての総合評価では、タスク成功率において従来手法比で25\%の向上を達成した。また、ユーザビリティテストにおいても、自然性と満足度の両面で有意な改善が確認された。

\chapter{考察}
\label{chap:discussion}

\section{結果の解釈}
\label{sec:interpretation}

実証分析の結果から、提案手法の有効性と特徴について考察する。RAG技術を統合した認知型決定モジュールは、従来のルールベースシステムと比較して、対話理解、状況認識、行動計画の各面で顕著な改善を示した。

特に注目すべき点は、システムが過去の経験から学習し、類似状況において適切な判断を下す能力を獲得したことである。これは、RAGの検索メカニズムが効果的に機能し、関連する過去の事例を適切に参照できていることを示している。

\section{結果の限界と問題点}
\label{sec:limitations}

本研究で提案したシステムには以下の限界と問題点が存在する。

まず、計算リソースの制約により、リアルタイム応答に課題がある。RAG技術の検索処理は時として処理時間が長く、即座の対応が求められる緊急時において遅延が生じる可能性がある。

また、多様な環境条件下での汎化性能についてはさらなる検証が必要である。実験環境は限定的であり、より複雑で予測困難な実環境での性能については追加の検証が求められる。

さらに、知識ベースの構築と維持には継続的な労力が必要であり、運用コストの観点からの課題も存在する。

\section{今後の課題}
\label{sec:future_work}

今後の研究では、以下の課題に取り組む必要がある。

システムの軽量化と高速化は最重要課題の一つである。エッジコンピューティング技術やモデル圧縮技術の活用により、リアルタイム性能の向上を図る必要がある。

多モーダル情報統合の改善も重要である。現在は主に言語情報に依存しているが、視覚情報や音響情報の統合により、より豊富な文脈理解が可能になると考えられる。

長期学習メカニズムの導入により、システムが運用中に継続的に改善される仕組みの構築も必要である。また、実際の商業環境での大規模展開に向けた実用性の向上も重要な課題である。

\chapter{結論}
\label{chap:conclusion}

\section{論文のまとめ}
\label{sec:summary}

本論文では、RAG技術と大規模言語モデルを統合した認知型決定モジュールを提案し、配膳ロボットの対話理解と緊急対応能力の向上を図った。

実証分析により、提案手法の有効性を確認し、従来手法と比較してタスク成功率で25\%、対話理解精度で18\%、緊急対応効率で35\%の改善を実現した。これらの結果は、RAG技術がロボットシステムの認知能力向上に有効であることを示している。

\section{研究成果の評価}
\label{sec:evaluation}

本研究の主要な貢献は以下の通りである:

第一に、RAG技術をロボット制御に応用する新たな枠組みの提案により、従来のルールベースシステムの限界を克服する道筋を示した。

第二に、対話履歴と環境情報を統合した認知型決定システムの実現により、より自然で適応的なロボット行動を可能にした。

第三に、実証実験による有効性の検証を通じて、提案手法の実用性を確認した。

これらの成果は、サービス型ロボットの知能化に重要な示唆を提供している。

\section{応用可能性と展望}
\label{sec:prospects}

提案手法は配膳ロボット以外のサービス型ロボットにも応用可能であり、医療支援ロボット、清掃ロボット、案内ロボットなどへの展開が期待される。

また、RAG技術の進歩と計算資源の向上により、より高度で実用的なシステムの実現が可能となるだろう。特に、5Gネットワークやエッジコンピューティングの普及により、リアルタイム性能の課題も解決されることが期待される。

将来的には、複数のロボットが協調して動作する分散型認知システムや、人間とロボットの自然な協働を実現するインターフェースの開発も可能になると考えられる。

\chapter*{謝辞}

本研究に際し、ご指導いただいた指導教員および共同研究者の皆様に深く感謝いたします。また、実験にご協力いただいた関係者の皆様にも心より御礼申し上げます。

\begin{thebibliography}{99}
  \raggedright

  \bibitem{zhu2024retrieval}
  Yichen Zhu, Zhicai Ou, Xiaofeng Mou, Jian Tang et al.
  \textit{Retrieval-Augmented Embodied Agents}.
  CVPR 2024.

  \bibitem{xie2024embodied}
  Quanting Xie, So Yeon Min, Pengliang Ji, Yonatan Bisk et al.
  \textit{Embodied-RAG: General Non-parametric Embodied Memory for Retrieval and Generation}.
  arXiv:2409.18313, 2024. \url{https://arxiv.org/abs/2409.18313}

  \bibitem{sarch2023helper}
  Gabriel Sarch, Yue Wu, Michael J. Tarr, Katerina Fragkiadaki.
  \textit{Open-Ended Instructable Embodied Agents with Memory-Augmented LLM (HELPER)}.
  EMNLP 2023.

  \bibitem{yoo2024exrap}
  Minjong Yoo, Jinwoo Jang, Wei-jin Park, Honguk Woo.
  \textit{Exploratory Retrieval-Augmented Planning for Continual Embodied Instruction Following (ExRAP)}.
  NeurIPS 2024.

  \bibitem{xu2024prag}
  Weiye Xu, Min Wang, Wengang Zhou, Houqiang Li.
  \textit{P-RAG: Progressive Retrieval-Augmented Generation for Embodied Everyday Task}.
  arXiv:2409.11279, 2024. \url{https://arxiv.org/abs/2409.11279}

\end{thebibliography}

\end{document}